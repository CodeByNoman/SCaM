% Options for packages loaded elsewhere
% Options for packages loaded elsewhere
\PassOptionsToPackage{unicode}{hyperref}
\PassOptionsToPackage{hyphens}{url}
\PassOptionsToPackage{dvipsnames,svgnames,x11names}{xcolor}
%
\documentclass[
  letterpaper,
  DIV=11,
  numbers=noendperiod]{scrartcl}
\usepackage{xcolor}
\usepackage{amsmath,amssymb}
\setcounter{secnumdepth}{5}
\usepackage{iftex}
\ifPDFTeX
  \usepackage[T1]{fontenc}
  \usepackage[utf8]{inputenc}
  \usepackage{textcomp} % provide euro and other symbols
\else % if luatex or xetex
  \usepackage{unicode-math} % this also loads fontspec
  \defaultfontfeatures{Scale=MatchLowercase}
  \defaultfontfeatures[\rmfamily]{Ligatures=TeX,Scale=1}
\fi
\usepackage{lmodern}
\ifPDFTeX\else
  % xetex/luatex font selection
\fi
% Use upquote if available, for straight quotes in verbatim environments
\IfFileExists{upquote.sty}{\usepackage{upquote}}{}
\IfFileExists{microtype.sty}{% use microtype if available
  \usepackage[]{microtype}
  \UseMicrotypeSet[protrusion]{basicmath} % disable protrusion for tt fonts
}{}
\makeatletter
\@ifundefined{KOMAClassName}{% if non-KOMA class
  \IfFileExists{parskip.sty}{%
    \usepackage{parskip}
  }{% else
    \setlength{\parindent}{0pt}
    \setlength{\parskip}{6pt plus 2pt minus 1pt}}
}{% if KOMA class
  \KOMAoptions{parskip=half}}
\makeatother
% Make \paragraph and \subparagraph free-standing
\makeatletter
\ifx\paragraph\undefined\else
  \let\oldparagraph\paragraph
  \renewcommand{\paragraph}{
    \@ifstar
      \xxxParagraphStar
      \xxxParagraphNoStar
  }
  \newcommand{\xxxParagraphStar}[1]{\oldparagraph*{#1}\mbox{}}
  \newcommand{\xxxParagraphNoStar}[1]{\oldparagraph{#1}\mbox{}}
\fi
\ifx\subparagraph\undefined\else
  \let\oldsubparagraph\subparagraph
  \renewcommand{\subparagraph}{
    \@ifstar
      \xxxSubParagraphStar
      \xxxSubParagraphNoStar
  }
  \newcommand{\xxxSubParagraphStar}[1]{\oldsubparagraph*{#1}\mbox{}}
  \newcommand{\xxxSubParagraphNoStar}[1]{\oldsubparagraph{#1}\mbox{}}
\fi
\makeatother

\usepackage{color}
\usepackage{fancyvrb}
\newcommand{\VerbBar}{|}
\newcommand{\VERB}{\Verb[commandchars=\\\{\}]}
\DefineVerbatimEnvironment{Highlighting}{Verbatim}{commandchars=\\\{\}}
% Add ',fontsize=\small' for more characters per line
\usepackage{framed}
\definecolor{shadecolor}{RGB}{241,243,245}
\newenvironment{Shaded}{\begin{snugshade}}{\end{snugshade}}
\newcommand{\AlertTok}[1]{\textcolor[rgb]{0.68,0.00,0.00}{#1}}
\newcommand{\AnnotationTok}[1]{\textcolor[rgb]{0.37,0.37,0.37}{#1}}
\newcommand{\AttributeTok}[1]{\textcolor[rgb]{0.40,0.45,0.13}{#1}}
\newcommand{\BaseNTok}[1]{\textcolor[rgb]{0.68,0.00,0.00}{#1}}
\newcommand{\BuiltInTok}[1]{\textcolor[rgb]{0.00,0.23,0.31}{#1}}
\newcommand{\CharTok}[1]{\textcolor[rgb]{0.13,0.47,0.30}{#1}}
\newcommand{\CommentTok}[1]{\textcolor[rgb]{0.37,0.37,0.37}{#1}}
\newcommand{\CommentVarTok}[1]{\textcolor[rgb]{0.37,0.37,0.37}{\textit{#1}}}
\newcommand{\ConstantTok}[1]{\textcolor[rgb]{0.56,0.35,0.01}{#1}}
\newcommand{\ControlFlowTok}[1]{\textcolor[rgb]{0.00,0.23,0.31}{\textbf{#1}}}
\newcommand{\DataTypeTok}[1]{\textcolor[rgb]{0.68,0.00,0.00}{#1}}
\newcommand{\DecValTok}[1]{\textcolor[rgb]{0.68,0.00,0.00}{#1}}
\newcommand{\DocumentationTok}[1]{\textcolor[rgb]{0.37,0.37,0.37}{\textit{#1}}}
\newcommand{\ErrorTok}[1]{\textcolor[rgb]{0.68,0.00,0.00}{#1}}
\newcommand{\ExtensionTok}[1]{\textcolor[rgb]{0.00,0.23,0.31}{#1}}
\newcommand{\FloatTok}[1]{\textcolor[rgb]{0.68,0.00,0.00}{#1}}
\newcommand{\FunctionTok}[1]{\textcolor[rgb]{0.28,0.35,0.67}{#1}}
\newcommand{\ImportTok}[1]{\textcolor[rgb]{0.00,0.46,0.62}{#1}}
\newcommand{\InformationTok}[1]{\textcolor[rgb]{0.37,0.37,0.37}{#1}}
\newcommand{\KeywordTok}[1]{\textcolor[rgb]{0.00,0.23,0.31}{\textbf{#1}}}
\newcommand{\NormalTok}[1]{\textcolor[rgb]{0.00,0.23,0.31}{#1}}
\newcommand{\OperatorTok}[1]{\textcolor[rgb]{0.37,0.37,0.37}{#1}}
\newcommand{\OtherTok}[1]{\textcolor[rgb]{0.00,0.23,0.31}{#1}}
\newcommand{\PreprocessorTok}[1]{\textcolor[rgb]{0.68,0.00,0.00}{#1}}
\newcommand{\RegionMarkerTok}[1]{\textcolor[rgb]{0.00,0.23,0.31}{#1}}
\newcommand{\SpecialCharTok}[1]{\textcolor[rgb]{0.37,0.37,0.37}{#1}}
\newcommand{\SpecialStringTok}[1]{\textcolor[rgb]{0.13,0.47,0.30}{#1}}
\newcommand{\StringTok}[1]{\textcolor[rgb]{0.13,0.47,0.30}{#1}}
\newcommand{\VariableTok}[1]{\textcolor[rgb]{0.07,0.07,0.07}{#1}}
\newcommand{\VerbatimStringTok}[1]{\textcolor[rgb]{0.13,0.47,0.30}{#1}}
\newcommand{\WarningTok}[1]{\textcolor[rgb]{0.37,0.37,0.37}{\textit{#1}}}

\usepackage{longtable,booktabs,array}
\usepackage{calc} % for calculating minipage widths
% Correct order of tables after \paragraph or \subparagraph
\usepackage{etoolbox}
\makeatletter
\patchcmd\longtable{\par}{\if@noskipsec\mbox{}\fi\par}{}{}
\makeatother
% Allow footnotes in longtable head/foot
\IfFileExists{footnotehyper.sty}{\usepackage{footnotehyper}}{\usepackage{footnote}}
\makesavenoteenv{longtable}
\usepackage{graphicx}
\makeatletter
\newsavebox\pandoc@box
\newcommand*\pandocbounded[1]{% scales image to fit in text height/width
  \sbox\pandoc@box{#1}%
  \Gscale@div\@tempa{\textheight}{\dimexpr\ht\pandoc@box+\dp\pandoc@box\relax}%
  \Gscale@div\@tempb{\linewidth}{\wd\pandoc@box}%
  \ifdim\@tempb\p@<\@tempa\p@\let\@tempa\@tempb\fi% select the smaller of both
  \ifdim\@tempa\p@<\p@\scalebox{\@tempa}{\usebox\pandoc@box}%
  \else\usebox{\pandoc@box}%
  \fi%
}
% Set default figure placement to htbp
\def\fps@figure{htbp}
\makeatother


% definitions for citeproc citations
\NewDocumentCommand\citeproctext{}{}
\NewDocumentCommand\citeproc{mm}{%
  \begingroup\def\citeproctext{#2}\cite{#1}\endgroup}
\makeatletter
 % allow citations to break across lines
 \let\@cite@ofmt\@firstofone
 % avoid brackets around text for \cite:
 \def\@biblabel#1{}
 \def\@cite#1#2{{#1\if@tempswa , #2\fi}}
\makeatother
\newlength{\cslhangindent}
\setlength{\cslhangindent}{1.5em}
\newlength{\csllabelwidth}
\setlength{\csllabelwidth}{3em}
\newenvironment{CSLReferences}[2] % #1 hanging-indent, #2 entry-spacing
 {\begin{list}{}{%
  \setlength{\itemindent}{0pt}
  \setlength{\leftmargin}{0pt}
  \setlength{\parsep}{0pt}
  % turn on hanging indent if param 1 is 1
  \ifodd #1
   \setlength{\leftmargin}{\cslhangindent}
   \setlength{\itemindent}{-1\cslhangindent}
  \fi
  % set entry spacing
  \setlength{\itemsep}{#2\baselineskip}}}
 {\end{list}}
\usepackage{calc}
\newcommand{\CSLBlock}[1]{\hfill\break\parbox[t]{\linewidth}{\strut\ignorespaces#1\strut}}
\newcommand{\CSLLeftMargin}[1]{\parbox[t]{\csllabelwidth}{\strut#1\strut}}
\newcommand{\CSLRightInline}[1]{\parbox[t]{\linewidth - \csllabelwidth}{\strut#1\strut}}
\newcommand{\CSLIndent}[1]{\hspace{\cslhangindent}#1}



\setlength{\emergencystretch}{3em} % prevent overfull lines

\providecommand{\tightlist}{%
  \setlength{\itemsep}{0pt}\setlength{\parskip}{0pt}}



 


\KOMAoption{captions}{tableheading}
\makeatletter
\@ifpackageloaded{caption}{}{\usepackage{caption}}
\AtBeginDocument{%
\ifdefined\contentsname
  \renewcommand*\contentsname{Table of contents}
\else
  \newcommand\contentsname{Table of contents}
\fi
\ifdefined\listfigurename
  \renewcommand*\listfigurename{List of Figures}
\else
  \newcommand\listfigurename{List of Figures}
\fi
\ifdefined\listtablename
  \renewcommand*\listtablename{List of Tables}
\else
  \newcommand\listtablename{List of Tables}
\fi
\ifdefined\figurename
  \renewcommand*\figurename{Figure}
\else
  \newcommand\figurename{Figure}
\fi
\ifdefined\tablename
  \renewcommand*\tablename{Table}
\else
  \newcommand\tablename{Table}
\fi
}
\@ifpackageloaded{float}{}{\usepackage{float}}
\floatstyle{ruled}
\@ifundefined{c@chapter}{\newfloat{codelisting}{h}{lop}}{\newfloat{codelisting}{h}{lop}[chapter]}
\floatname{codelisting}{Listing}
\newcommand*\listoflistings{\listof{codelisting}{List of Listings}}
\makeatother
\makeatletter
\makeatother
\makeatletter
\@ifpackageloaded{caption}{}{\usepackage{caption}}
\@ifpackageloaded{subcaption}{}{\usepackage{subcaption}}
\makeatother
\usepackage{bookmark}
\IfFileExists{xurl.sty}{\usepackage{xurl}}{} % add URL line breaks if available
\urlstyle{same}
\hypersetup{
  pdftitle={Soil Texture Classification and Spatial Mapping},
  pdfauthor={Noman Ahmad},
  colorlinks=true,
  linkcolor={blue},
  filecolor={Maroon},
  citecolor={Blue},
  urlcolor={Blue},
  pdfcreator={LaTeX via pandoc}}


\title{Soil Texture Classification and Spatial Mapping}
\author{Noman Ahmad}
\date{2025-11-24}
\begin{document}
\maketitle

\renewcommand*\contentsname{Table of contents}
{
\hypersetup{linkcolor=}
\setcounter{tocdepth}{3}
\tableofcontents
}

\newpage

\section{Introduction}\label{introduction}

Soil texture, defined by the relative proportions of sand, silt, and
clay, is a fundamental property that influences water retention,
nutrient availability, erosion susceptibility, and overall soil
fertility (Daniels 2016). Understanding the spatial distribution of soil
texture is essential for applications in agriculture, land management,
hydrology, and environmental modeling. Field-based soil sampling
provides precise measurements at specific locations but is typically
limited in spatial coverage due to time, labor, and cost constraints
(McBratney, Mendonça Santos, and Minasny 2003). On the other hand,
global soil datasets, such as the World Soil Grid (WSG), provide
spatially continuous estimates of soil properties at continental to
global scales (Hengl et al. 2017). While useful for broad-scale
analyses, these datasets may not accurately capture fine-scale
variability present in local landscapes.

Integrating field-measured soil data with global soil grids, combined
with spatial interpolation techniques such as Inverse Distance Weighting
(IDW) and Kriging, allows researchers to generate detailed maps of soil
texture and identify patterns of spatial heterogeneity (Li et al. 2011).
Such integration improves the reliability of soil property assessments
and supports decision-making in precision agriculture and environmental
management. The present study focuses on classifying soil texture from
field samples using the USDA soil texture triangle, generating spatially
continuous interpolated maps, and comparing these results with World
Soil Grid estimates. The analysis aims to highlight areas of agreement
and discrepancy between local measurements and global predictions,
thereby providing a more comprehensive understanding of soil texture
variability in the study area.

\subsection{Research Question}\label{research-question}

How do soil texture classes derived from field-measured soil samples
differ from those estimated by the World Soil Grid, and what spatial
patterns emerge when using interpolation techniques in R?

\subsection{Hypothesis}\label{hypothesis}

Field-derived soil texture classes will show finer spatial variation
than World Soil Grid predictions, with higher disagreement in areas of
micro-scale heterogeneity.

\subsection{Objective}\label{objective}

To classify soil texture using the USDA texture triangle, map spatial
variability using interpolation, and compare results with World Soil
Grid data.

\section{\texorpdfstring{\textbf{Data
Sources}}{Data Sources}}\label{data-sources}

\textbf{1. Field Soil Data}

\begin{itemize}
\item
  Sand (\%), Silt (\%), Clay (\%)
\item
  Coordinates (WGS84)
\item
  Format: CSV / XLSX
\end{itemize}

\textbf{2. World Soil Grid (ISRIC)}

\begin{itemize}
\item
  Sand, Silt, Clay (0-5 cm depth)
\item
  Resolution: 250 m
\item
  Format: GeoTIFF
\end{itemize}

\textbf{3. Study Area}

\begin{itemize}
\tightlist
\item
  Shapefile defining the project boundary
\end{itemize}

\section{Analysis}\label{analysis}

\subsection{Requirements}\label{requirements}

\begin{Shaded}
\begin{Highlighting}[]
\FunctionTok{library}\NormalTok{(tidyverse)}
\FunctionTok{library}\NormalTok{(readxl)}
\FunctionTok{library}\NormalTok{(soiltexture)}
\end{Highlighting}
\end{Shaded}

\subsection{Data Cleaning}\label{data-cleaning}

\begin{Shaded}
\begin{Highlighting}[]
\CommentTok{\# 1. Read raw data}
\NormalTok{soil }\OtherTok{\textless{}{-}} \FunctionTok{read\_xlsx}\NormalTok{(}\StringTok{"A:/Class\_Assignments/SCaM/Data/Raw/Soil\_Water\_Data.xlsx"}\NormalTok{)}
\FunctionTok{print}\NormalTok{(soil)}

\CommentTok{\# 2. Columns rename for clarity}
\NormalTok{soil }\OtherTok{\textless{}{-}}\NormalTok{ soil }\SpecialCharTok{\%\textgreater{}\%}
  \FunctionTok{rename}\NormalTok{(}\AttributeTok{SAND =} \StringTok{\textasciigrave{}}\AttributeTok{Sand\_\%}\StringTok{\textasciigrave{}}\NormalTok{, }\AttributeTok{SILT =} \StringTok{\textasciigrave{}}\AttributeTok{Silt\_\%}\StringTok{\textasciigrave{}}\NormalTok{, }\AttributeTok{CLAY =} \StringTok{\textasciigrave{}}\AttributeTok{Clay\_\%}\StringTok{\textasciigrave{}}\NormalTok{, }
         \AttributeTok{OM =} \StringTok{\textasciigrave{}}\AttributeTok{O.M}\StringTok{\textasciigrave{}}\NormalTok{, }\AttributeTok{SS =} \StringTok{\textasciigrave{}}\AttributeTok{Soil\_Saturation\_\%}\StringTok{\textasciigrave{}}\NormalTok{)}

\CommentTok{\# 3. Missing values removal}
\NormalTok{soil }\OtherTok{\textless{}{-}}\NormalTok{ soil }\SpecialCharTok{\%\textgreater{}\%}
  \FunctionTok{filter}\NormalTok{(}\SpecialCharTok{!}\FunctionTok{is.na}\NormalTok{(SAND) }\SpecialCharTok{\&} \SpecialCharTok{!}\FunctionTok{is.na}\NormalTok{(SILT) }\SpecialCharTok{\&} \SpecialCharTok{!}\FunctionTok{is.na}\NormalTok{(CLAY) }
         \SpecialCharTok{\&} \SpecialCharTok{!}\FunctionTok{is.na}\NormalTok{(OM) }\SpecialCharTok{\&} \SpecialCharTok{!}\FunctionTok{is.na}\NormalTok{(SS))}

\CommentTok{\# 4. Validation of sum of soil texture components }
\NormalTok{soil }\OtherTok{\textless{}{-}}\NormalTok{ soil }\SpecialCharTok{\%\textgreater{}\%}
  \FunctionTok{mutate}\NormalTok{(}\AttributeTok{total =}\NormalTok{ SAND }\SpecialCharTok{+}\NormalTok{ SILT }\SpecialCharTok{+}\NormalTok{ CLAY) }\SpecialCharTok{\%\textgreater{}\%}
  \FunctionTok{filter}\NormalTok{(total }\SpecialCharTok{==} \DecValTok{100}\NormalTok{ )}

\CommentTok{\# 5. Save cleaned \& seperated soil texture data}
\NormalTok{ST.data }\OtherTok{\textless{}{-}}\NormalTok{ soil }\SpecialCharTok{\%\textgreater{}\%} \FunctionTok{select}\NormalTok{(Address\_code, SAND, SILT, CLAY, OM, SS)}
\FunctionTok{write\_csv}\NormalTok{(ST.data, }\StringTok{"Data/Processed/Soil\_Data\_Cleaned.csv"}\NormalTok{)}
\end{Highlighting}
\end{Shaded}

\subsection{Texture Classification}\label{texture-classification}

\begin{Shaded}
\begin{Highlighting}[]
\FunctionTok{library}\NormalTok{(soiltexture)}

\CommentTok{\# 1. Read processed data}
\NormalTok{soil\_data }\OtherTok{\textless{}{-}} \FunctionTok{read.csv}\NormalTok{(}\StringTok{"Data/Processed/Soil\_Data\_Cleaned.csv"}\NormalTok{)}

\CommentTok{\# 2. Get ISSS/USDA/FAO texture matrix}
\NormalTok{tex\_matrix }\OtherTok{\textless{}{-}} \FunctionTok{TT.points.in.classes}\NormalTok{(}\AttributeTok{tri.data =}\NormalTok{ soil\_data[}
  \FunctionTok{c}\NormalTok{(}\StringTok{"SAND"}\NormalTok{, }\StringTok{"SILT"}\NormalTok{, }\StringTok{"CLAY"}\NormalTok{)], }\AttributeTok{class.sys =} \StringTok{"USDA.TT"}\NormalTok{)}
\FunctionTok{print}\NormalTok{(tex\_matrix)}

\CommentTok{\# 3. Convert matrix (0/1) to a single texture class column}
\NormalTok{soil\_data}\SpecialCharTok{$}\NormalTok{Texture\_Class }\OtherTok{\textless{}{-}} \FunctionTok{apply}\NormalTok{(tex\_matrix, }\DecValTok{1}\NormalTok{, }\ControlFlowTok{function}\NormalTok{(row) }
\NormalTok{  \{classes }\OtherTok{\textless{}{-}} \FunctionTok{names}\NormalTok{(row)[row }\SpecialCharTok{==} \DecValTok{1}\NormalTok{]; }\ControlFlowTok{if}\NormalTok{ (}\FunctionTok{length}\NormalTok{(classes) }\SpecialCharTok{==} \DecValTok{0}\NormalTok{) }
    \FunctionTok{return}\NormalTok{(}\ConstantTok{NA}\NormalTok{); classes[}\DecValTok{1}\NormalTok{]\})}
\NormalTok{SoilTC }\OtherTok{\textless{}{-}}\NormalTok{ soil\_data }\SpecialCharTok{\%\textgreater{}\%} \FunctionTok{select}\NormalTok{(Address\_code, SAND, SILT, CLAY, Texture\_Class)}
\FunctionTok{print}\NormalTok{(SoilTC)}
\FunctionTok{write.csv}\NormalTok{(SoilTC, }\StringTok{"A:/Class\_Assignments/SCaM/Outputs/Tables/Soiltexc.csv"}\NormalTok{)}

\CommentTok{\# 4. Soil Texture Triangle}

\FunctionTok{tiff}\NormalTok{(}\StringTok{"A:/Class\_Assignments/SCaM/Outputs/Figures/Soil\_Texture\_Triangle.tiff"}\NormalTok{,}
     \AttributeTok{width =} \DecValTok{2000}\NormalTok{, }\AttributeTok{height =} \DecValTok{1800}\NormalTok{, }\AttributeTok{res =} \DecValTok{300}\NormalTok{, }\AttributeTok{compression =} \StringTok{"lzw"}\NormalTok{)}

\FunctionTok{par}\NormalTok{(}\AttributeTok{family =} \StringTok{"serif"}\NormalTok{)}

\NormalTok{classes }\OtherTok{\textless{}{-}} \FunctionTok{unique}\NormalTok{(soil\_data}\SpecialCharTok{$}\NormalTok{Texture\_Class)}
\NormalTok{classes }\OtherTok{\textless{}{-}}\NormalTok{ classes[}\SpecialCharTok{!}\FunctionTok{is.na}\NormalTok{(classes)]}

\NormalTok{palette\_colors }\OtherTok{\textless{}{-}} \FunctionTok{rainbow}\NormalTok{(}\FunctionTok{length}\NormalTok{(classes))}
\FunctionTok{names}\NormalTok{(palette\_colors) }\OtherTok{\textless{}{-}}\NormalTok{ classes}

\FunctionTok{TT.plot}\NormalTok{(}\AttributeTok{class.sys =} \StringTok{"USDA.TT"}\NormalTok{,  }\AttributeTok{main =} \StringTok{"USDA Soil Texture Classification"}\NormalTok{,}
  \AttributeTok{cex.lab =} \FloatTok{1.1}\NormalTok{, }\AttributeTok{cex.axis =} \FloatTok{1.0}\NormalTok{, }\AttributeTok{cex.main =} \FloatTok{1.25}\NormalTok{, }\AttributeTok{frame.bg.col =} \StringTok{"white"}\NormalTok{)}

\FunctionTok{TT.points}\NormalTok{(}\AttributeTok{tri.data =}\NormalTok{ soil\_data, }\AttributeTok{geo =} \FunctionTok{TT.geo.get}\NormalTok{(}\StringTok{"USDA.TT"}\NormalTok{),}
  \AttributeTok{col =}\NormalTok{ palette\_colors[soil\_data}\SpecialCharTok{$}\NormalTok{Texture\_Class], }\AttributeTok{pch =} \DecValTok{19}\NormalTok{, }\AttributeTok{cex =} \FloatTok{1.2}\NormalTok{)}

\FunctionTok{legend}\NormalTok{(}\StringTok{"topleft"}\NormalTok{, }\AttributeTok{inset =} \FunctionTok{c}\NormalTok{(}\FloatTok{1.04}\NormalTok{, }\FloatTok{0.06}\NormalTok{), }\AttributeTok{legend =}\NormalTok{ classes, }
       \AttributeTok{col =}\NormalTok{ palette\_colors[classes], }\AttributeTok{title =} \StringTok{"Classes"}\NormalTok{, }\AttributeTok{title.cex =} \FloatTok{1.0}\NormalTok{, }
       \AttributeTok{title.font =} \DecValTok{2}\NormalTok{, }\AttributeTok{pch =} \DecValTok{19}\NormalTok{, }\AttributeTok{pt.cex =} \FloatTok{1.0}\NormalTok{, }\AttributeTok{cex =} \FloatTok{0.9}\NormalTok{, }\AttributeTok{bty =} \StringTok{"o"}\NormalTok{)}

\FunctionTok{dev.off}\NormalTok{()}
\end{Highlighting}
\end{Shaded}

\begin{figure}[H]

{\centering \includegraphics[width=5.80208in,height=\textheight,keepaspectratio]{images/Soil_Texture_Triangle.jpg}

}

\caption{USDA Soil Texture Triangle showing the classification of soil
texture based on the proportions of sand, silt, and clay.}

\end{figure}%

\begin{center}\rule{0.5\linewidth}{0.5pt}\end{center}

\begin{center}
TO BE CONTINUED
\end{center}

\begin{center}\rule{0.5\linewidth}{0.5pt}\end{center}

\newpage

\section*{References}\label{references}
\addcontentsline{toc}{section}{References}

\phantomsection\label{refs}
\begin{CSLReferences}{1}{0}
\bibitem[\citeproctext]{ref-daniels2016}
Daniels, W. Lee. 2016. {``The Nature and Properties of Soils, 15th
EditionRay R.Weil and Nyle C.Brady. Pearson Press, Upper Saddle River
NJ, 2017. 1086 p. {\$}164.80. ISBN-10: 0-13-325448-8; ISBN-13:
978-0-13-325448-8. Also Available as eText for {\$}67.99.''} \emph{Soil
Science Society of America Journal} 80 (5): 1428--28.
\url{https://doi.org/10.2136/sssaj2016.0005br}.

\bibitem[\citeproctext]{ref-hengl2017}
Hengl, Tomislav, Jorge Mendes de Jesus, Gerard B. M. Heuvelink, Maria
Ruiperez Gonzalez, Milan Kilibarda, Aleksandar Blagotić, Wei Shangguan,
et al. 2017. {``SoilGrids250m: Global Gridded Soil Information Based on
Machine Learning.''} Edited by Ben Bond-Lamberty. \emph{PLOS ONE} 12
(2): e0169748. \url{https://doi.org/10.1371/journal.pone.0169748}.

\bibitem[\citeproctext]{ref-li2011}
Li, Jin, Andrew D. Heap, Anna Potter, and James J. Daniell. 2011.
{``Application of Machine Learning Methods to Spatial Interpolation of
Environmental Variables.''} \emph{Environmental Modelling \& Software}
26 (12): 1647--59. \url{https://doi.org/10.1016/j.envsoft.2011.07.004}.

\bibitem[\citeproctext]{ref-mcbratney2003}
McBratney, A. B, M. L Mendonça Santos, and B Minasny. 2003. {``On
Digital Soil Mapping.''} \emph{Geoderma} 117 (1-2): 3--52.
\url{https://doi.org/10.1016/s0016-7061(03)00223-4}.

\end{CSLReferences}




\end{document}
